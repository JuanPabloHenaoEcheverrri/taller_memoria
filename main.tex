\documentclass{article}
\usepackage[utf8]{inputenc}
\usepackage[spanish]{babel}
\usepackage{listings}
\usepackage{graphicx}
\graphicspath{ {images/} }
\usepackage{cite}

\begin{document}

\begin{titlepage}
    \begin{center}
        \vspace*{1cm}
            
        \Huge
        \textbf{Título del trabajo}
            
        \vspace{0.5cm}
        \LARGE
        Subtítulo
            
        \vspace{1.5cm}
            
        \textbf{Nombres y Apellidos del autor}
            
        \vfill
            
        \vspace{0.8cm}
            
        \Large
        Despartamento de Ingeniería Electrónica y Telecomunicaciones\\
        Universidad de Antioquia\\
        Medellín\\
        Septiembre de 2020
            
    \end{center}
\end{titlepage}

\tableofcontents

\section{Introducción}
Esta es la primera sección, podemos agregar algunos elementos adicionales y todo será escrito correctamente. Más aún, si una palabra es demasiado larga y tiene que ser truncada, babel tratará de truncarla correctamente dependiendo del idioma.

\section{Contenido} \label{contenido}

\subsection{Memoria de computador}

Una memorias cumplen un papel muy importante en un computador y su funcionamiento, ya que es el el dispositivio donde se alamacena temporalmente toda la información con la que trabajan los microprocesadores para procesarla y devolver los resultados que los usuarios requieren. \cite{taller} \\

En terminos dela arquitectura de una memoria de computador se puede denominar como un conjunto de celdas junto con los circuitos asociados que se necesitan para ingresar y sacar información de almacenamiento. Una palabra de memoria es un grupo de 1 y 0, que pueder representar un número un codico de istrucción, uno o mas caracteres alfanúmericos, o cualquier otra información en código binario. \cite{ingenieu} \\

\subsection{Tipo de memorias que conozco}
\textbf{Memoria Rom:} Tambien conocida como memoria de solo lectura, es una memoria la cual almacena información pero esta no puede ser destruida ni reprogramable. \\

\textbf{Memoria Ram:} A difrenecia de la memoria ROM, la información guardada en una memoria RAM se puede reeprogramar o reemplazar. Su función es la de almcenar instrucciones o datos, para evitar que estos tengan que pasar nuevamente por el procesador. \\

\subsection{Como se gestiona la memoria en un computador.}
Para entender como funciona una memoria, es importante usar la siguiente analogía: Hay que imaginar que un trabajador de realizar una serie de tareas contables. El cajon donde guardan todos los documentos podria ser el disco duro; los documentos son equivalentes a los datos y a la información a procesar; la mesa de trabajo o el escritorio donde se apilan dichos archivos sería el equivalente a la memoria donde se almacena temporalmente la información que se encuentra en procesamiento; mientras la persona seria el microprocesador que realiza diferentes tareas.
\begin{itemize}
    \item Primero se sacan del cajón (disco duro) los documentos administrativos y se les lleva a un escritorio (memoria) donde se apilan para poder trabajar.
    \item Se toma el primer archivo de la pila para que el empleado (microprocesador) realice los calculos necesarios, así como otras tareas y finalmente se ingresen las modificaciones y los resultados de datos procesados en dicho documento.
    \item Se regresa dicho documento procesado a otra parte del escritorio (memoria) donde se colocarán los documentos procesados.
    \item Luego se toma otro documento no procesados y se repiten los dos pasos anteriores.Eso se reitera una y otra vez hasta que todos los documentos hayan sido procesados. 
    \item Finalmente cuando se terminan de procesar todos los documentos, los cuales se encuentran apilados en la parte del escritorio (memoria) de documentos ya procesados, se toman y se vuelven a guardar en el cajón (disco duro) de almacenamiento de archivos. 
  \end{itemize}  
    Es importante recalcar que el archivo nunca se elimina del disco duro, simplemente se crea una copia en este, y despues es reemplazado por lo que haya en la memoria. \cite{taller}

\subsection{¿Que hace que una memoria sea mas rápida que otra? ¿Por qué esto es importante?. }













\section{Conclusión} \label{conclulsion}

\bibliographystyle{IEEEtran}
\bibliography{references}

\end{document}
